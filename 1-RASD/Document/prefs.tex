\usepackage[margin=3cm]{geometry}
\usepackage[english]{babel}
\usepackage[T1]{fontenc}
\usepackage[latin1]{inputenc}
\usepackage{graphicx}
\usepackage{titlesec, blindtext, color}
\usepackage{fancyhdr}
\usepackage{hyperref}
\usepackage{float}
\usepackage{enumitem}
\usepackage{lastpage}
\usepackage{listings}
\usepackage{color}
\usepackage{amsmath}

%----------------------------------------------------------------------------------------
%   Title Style Chapter
%----------------------------------------------------------------------------------------
\definecolor{gray75}{gray}{0.70}
\newcommand{\hsp}{\hspace{15pt}}
\newcommand\tab[1][1cm]{\hspace*{#1}}
\newcommand{\forceindent}{\leavevmode{\parindent=1em\indent}}
\titleformat{\chapter}[hang]{\Huge\bfseries}{\thechapter\hsp\textcolor{gray75}{|}\hsp}{0pt}{\Huge\bfseries}


%----------------------------------------------------------------------------------------
%  ToC
%----------------------------------------------------------------------------------------
\hypersetup{
    colorlinks,
    citecolor=black,
    filecolor=black,
    linkcolor=black,
    urlcolor=black
}

% Renaming of ToC
\addto\captionsenglish{% Replace "english" with the language you use
  \renewcommand{\contentsname}%
    {Table of Contents}%
}

%----------------------------------------------------------------------------------------
%   Header/Footer
%----------------------------------------------------------------------------------------
\pagestyle{fancy} 
\fancyhf{}
%\renewcommand{\chaptermark}[1]{\markboth{#1}{}}

%\renewcommand{\chaptermark}[1]{%
% \markboth{\chaptername
% \ \thechapter.\ #1}{}}

%\renewcommand{\sectionmark}[1]{\markright{\thesection.\ #1}}

\lhead{\nouppercase{\leftmark}}
%\chead{}
\rhead{\nouppercase{\rightmark}}
\lfoot{RASD - \textit{Version 1.0}}
%\cfoot{}
\rfoot{\thepage}
\renewcommand{\headrulewidth}{0.008pt}
\renewcommand{\footrulewidth}{0.008pt}

% Redefine the plain page style
\fancypagestyle{plain}{%
	\fancyhf{}%
	\fancyfoot[R]{\thepage}%
	\renewcommand{\headrulewidth}{0pt}% Line at the header invisible
	\renewcommand{\footrulewidth}{0.008pt}% Line at the footer visible
}

%-----------
% New Page for each section
%---------
%\newcommand{\sectionbreak}{\clearpage}

%-------
% Numbered list for goals
%-------
\newlist{Goal}{enumerate}{1}
\setlist[Goal]{label=[G\arabic*],leftmargin=0.6in}

%-------
% Numbered list for requirements
%-------
\newlist{Req}{enumerate}{1}
\setlist[Req]{label=[R\arabic*],leftmargin=0.6in}

%-------
% Numbered list for domain assumptions
%-------
\newlist{DA}{enumerate}{1}
\setlist[DA]{label=[DA\arabic*],leftmargin=0.6in}

%-------
% Numbered list for text assumptions
%-------
\newlist{TA}{enumerate}{1}
\setlist[TA]{label=[TA\arabic*],leftmargin=0.6in}

%-----
% Image box, arguments: file path, caption
%----
\newcommand{\imagebox}[3]{
\begin{figure}[H]
\centering
\fbox{\includegraphics[width=\textwidth,keepaspectratio,angle=#3]{#1}}
\caption{#2}
\end{figure}
}

%---
% Alloy listing
%---
\definecolor{pgreen}{rgb}{0,0.64,0.013}
\definecolor{pblue}{rgb}{0.13,0.13,1}

\lstdefinelanguage{alloy}{
	keywords={
		assert, pred, all, no, lone, one, some, check, run,
		but, let, implies, not, iff, in, and, or, set, sig, Int, int,
		if, then, else, exactly, disj, fact, fun, module, abstract,
		extends, open, none, univ, iden, seq},
	literate=%
	{:}{$\colon$}1
	%{|}{$\bullet$}1
	{==}{$=$}1
	{=}{$=$}1
	{!=}{$\neq$}1
	{&&}{$\land$}1
	{||}{$\lor$}1
	{<=}{$\le$}1
	{>=}{$\ge$}1
	%{all}{$\forall$}1
	{exists}{$\exists$}1
	{!in}{$\not\in$}1
	{\\in}{$\in$}1
	{=>}{$\implies$}3
	% the following isn't actually Alloy, but it gives the option to produce nicer latex
	{|=>}{$\Rightarrow$}2
	{<=set}{$\subseteq$}1
	{+set}{$\cup$}1
	{*set}{$\cap$}1
	{==>}{$\Longrightarrow$}3
	{<=>}{$\iff$}4
	{...}{$\ldots$}1
	{\\hl}{$\hline$}1
	{\\alpha}{$\alpha$}1
	{\\beta}{$\beta$}1
	{\\gamma}{$\gamma$}1
	{\\delta}{$\delta$}1
	{\\epsilon}{$\epsilon$}1
	{\\zeta}{$\zeta$}1
	{\\eta}{$\eta$}1
	{\\theta}{$\theta$}1
	{\\iota}{$\iota$}1
	{\\kappa}{$\kappa$}1
	{\\lambda}{$\lambda$}1
	{\\mu}{$\mu$}1
	{\\nu}{$\nu$}1
	{\\xi}{$\xi$}1
	{\\pi}{$\pi$}1
	{\\rho}{$\rho$}1
	{\\sigma}{$\sigma$}1
	{\\tau}{$\tau$}1
	{\\upsilon}{$\upsilon$}1
	{\\phi}{$\phi$}1
	{\\chi}{$\chi$}1
	{\\psi}{$\psi$}1
	{\\omega}{$\omega$}1
	{\\Gamma}{$\Gamma$}1
	{\\Delta}{$\Delta$}1
	{\\Theta}{$\Theta$}1
	{\\Lambda}{$\Lambda$}1
	{\\Xi}{$\Xi$}1
	{\\Pi}{$\Pi$}1
	{\\Sigma}{$\Sigma$}1
	{\\Upsilon}{$\Upsilon$}1
	{\\Phi}{$\Phi$}1
	{\\Psi}{$\Psi$}1
	{\\Omega}{$\Omega$}1
	{\\EOF}{\;}1,
	sensitive=true,
	morecomment=[l]//,%
	%morecomment=[l]{--},%
	morecomment=[s]{/*}{*/},%
	morestring=[b]",
	numbers=none,
	firstnumber=1,
	%numberstyle=\tiny,
	%basicstyle=\scriptsize\ttfamily,
	commentstyle=\color{pgreen},
	keywordstyle=\color{pblue},
	%ndkeywordstyle=\bfseries,
}

\lstnewenvironment{alloyCode}{%	
	\lstset{language=alloy,
		floatplacement={tbp},
		captionpos=b,
		xleftmargin=8pt,
		xrightmargin=8pt,
		 breaklines=true,
		  tabsize=3,
		%basicstyle=\ttfamily,
	}
} {}
%--- End of listing for Alloy
%
%---
% Recall item commands
%---
\newcommand\itemtext[2]{%
	\expandafter\gdef\csname item#1\endcsname{#2}%
	\label{#1}#2}

\newcommand\useitemtext[2]{\csname item#1\endcsname}
