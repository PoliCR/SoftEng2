\documentclass[11pt,a4paper]{report}

%INPUT of SETTINGS (usepackage, styles, new command,...)
\input{dd_prefs}

%-------------------------------------------------------------------------
%   START DOCUMENT 
%-------------------------------------------------------------------------
\begin{document}

% INPUT OF TITLE PAGE
\input{dd_titlepage}

%-------------------------------------------------------------------------
%   TOC
%-------------------------------------------------------------------------
%\pagestyle{plain}
\pagenumbering{roman}
\thispagestyle{empty}
\tableofcontents
\listoffigures
\cleardoublepage
\pagenumbering{arabic}
\pagestyle{fancy}
%--------------------------------------------------------------------
%  BEGIN TEXT
%--------------------------------------------------------------------
\chapter{Introduction}
\section{Purpose}
The current document describes the architecture of the system \textit{PowerEnJoy}. It covers both the physical implementation with an insight on the deployment layout and the logical distribution of the software modules.\\The \textit{Design Document} takes into account all the considerations that have been made in the previous document, the \textit{RASD}, and shows how these issues can be tackled in concrete with design choices. 

\section{Scope}
The \textit{Design Document} explores the architecture of the system-to-be by means of software design principles and known patterns that fit the given problem. Throughout the document the system will be analysed at different conceptual and granular levels with heterogeneous views and diagrams: the Component view, the Deployment view, the Runtime view with sequence diagrams, the User Interfaces and the ER Diagram. Moreover, the most significant algorithms of the application will be sketched up. The document is also a point of reference for the traceability of the requirements that have been identified in the \textit{RASD}: a dedicated section will explain how these requirements are fulfilled with the proposed architecture elements.

\section{Definitions, Acronyms and Abbreviations}
\begin{description}
\item[PowerEnJoy]is the name of the system that has to be developed.
\item[System]sometimes called also \textit{system-to-be}, represents the application that will be described and implemented.
In particular, its structure and implementation will be explained in the following documents. People that will use the car-sharing service will interact with it, via some interfaces, in order to complete some operations (e.g.: reservation and renting).
\item[Renting]it is the act of picking-up an available car and of starting to drive.
\item[Ride] the event of picking-up a car, driving through the city and parking it. Every Ride is associated to a single user and to a single car.
\item[Reservation]it is the action of booking an available car.
\item[Car] a car is an electrical vehicle that will be used by a registered user.
\item[Not Registered User] indicates a person who hasn't registered to the system yet; for this reason he can't access to any of the offered function. The only possible action that he can carry out is the registration to get a personal account.
\item[Registered User] interacts with the system to use the sharing service. He has an account (which contains personal information, driving license number and payment data) that must be used to access to the application in order to exploit all the functionalities.
\item[Employee] it's a person who works for the company, whose main task is to plug into the power grid those cars that haven't been plugged in by the users. He is also in charge of taking care of the status of the cars and of moving the vehicles from a safe area to a charging area and vice versa if needed.
\item[Safe Area] indicates a set of parking lots where the users have to leave the car at the end of the rent; the set of the Safe Areas is pre-defined by the system management. These areas are spread all over the city.
\item[Plug] defines the electrical component that physically connects the car to the power grid.
\item[Charging Area] is a special \textit{Safe Area} that also provides a certain number of plugs that connect the cars to the power grid in order to recharge the battery.
\item[Registration] the procedure that an unregistered user has to perform to become a registered user. At the end, the unregistered user will have an account. To complete this operation three different types of data are required: personal information, driving license number and payment info.
\item[Search] this functionality lets the registered user search for available cars within a certain range from his/her current position or from a specified address.
\item[RASD] is the acronym of \textit{Requirements Analysis and Specification Document}
\item[DD] is the acronym of \textit{Design Document}
\end{description}

\section{Reference Documents}
During the writing of this document, the following resources have been taken into account:
\begin{itemize}
	\item Specification Document:
	\begin{itemize}
		\item \textit{Assignments+AA+2016-2017.pdf}
		\item \textit{RASD.pdf}
	\end{itemize}
	\item Example document:
	\begin{itemize}
		\item \textit{Sample Design Deliverable Discussed on Nov. 2.pdf}
	\end{itemize}
	\item Papers on \textit{Green Move Project}
\end{itemize}
\section{Document Structure}
The first chapter recalls the purpose and the scope of the system that has to be developed and in particular of the current document, which focuses on the architecture of \textit{PowerEnJoy}.\\
In the second chapter, which is the core of the \textit{Design Document},
the overall architecture will be presented at various levels of abstraction. After that a brief explanation of the architectural styles and patterns used is provided.\\
The third chapter covers the algorithmic part of the system; some interesting algorithms that will be useful later on in the real implementation are described in pseudocode.\\
The fourth chapter contains the mockups of both the webapp user interfaces and the ad-hoc touch screen embedded into the cars. Along with them, there is a small description.\\
The fifth chapter reports the mapping of the requirements, identified in the \textit{RASD}, with respect to the architecture's components.\\
The sixth and seventh chapters simply refers to the time effort spent in writing the \textit{DD} and the tools used to do it.
\chapter{Architectural Design}
\section{Overview}
%high level components and their interaction
% here you can introduce the high level components of your architecture
%and describe the main interaction between them (no details here. You can say why some components talk to each other, why, if the communication is synchronous or asynchronous, any other info you think is useful at this point).
The system-to-be, \textit{PowerEnJoy}, will adopt a three-tier architecture, which means that the application will be designed by taking into account three physical levels as shown in the following figure:
\imageboxNoFit{../Architecture_Overview_Img/DD_a}{High Level Architecture}{0}{0.6}
\begin{itemize}
\item \textbf{Client Tier} includes all the devices (PCs, smartphones and car screens) and software with which the people (\textit{Users} and \textit{Employees}) and the cars will be able to interact with the 			system.
\item \textbf{Application Tier} represents the set of machines (and 				software) where the core of the application will be run on.
\item \textbf{Data Tier} is the physical layer in charge of storing the data that are necessary to the system.
\end{itemize}
\noindent Each of these tiers will run a precise piece of software and the following sections will go into the details in terms of software components, but it is useful to have already at this point a glimpse on the logical distribution of the application modules.\\
\imageboxNoFit{../Architecture_Overview_Img/DD_b}{Overview Architecture}{0}{0.6}
\noindent The figure suggests that \textit{Client Tier} will be organized quite differently between the PCs, smartphones and the Car sides. While PCs in the Client Tier will only host the Presentation layer, in the Car there will be both the GUI and a small part of application logic. This is necessary to elaborate some information on the car, such as the interaction of the car with other components within the car itself (i.e. sensors) or out of it (i.e. the charging plug).\\
For both the PCs and the Car, however, the business logic of the application is on an Application server, in the Web Tier. Finally, the persistent data are stored in the Data Tier.

\section{Component View}
%here you have a refinement of what you have in Section 4.B and identify sub-components. For instance, the diagram in slide 6 could be a diagram showing a  component view
This section focuses on the internal structure of the system-to-be in terms of the sub-components that it will be composed of and their relations.\\Before going into details, it is useful to recall from the Overview section that the deployment of the application is slightly different between the PC/Smartphone and the Car platforms. This is reflected in the following component diagram in which the \textit{Device} (i.e. Smartphone and PC) and the \textit{Car} nodes are shaped in different ways, hosting different components and different interfaces with the server.\\One one hand, the \textit{Devices} interact with the application through \textit{WebAppAPI}, on the other hand, the \textit{Car} exposes its proper \textit{CarAPI} and deals with the \textit{ServerAPI} to share information with the application running on the server.\\With regard to the Webapp, there are the following components to take into account:
\begin{description}
\item[ViewRender]is the component in charge of managing the viewable part of the application. It has two main roles: firstly, to fetch the user inputs coming from the Device browser (through the \textit{WebAppAPI}) and to pass them to the \textit{Dispatcher}, and secondly to ship the Web pages that it receives from the other components to the \textit{Device} browser.
\item[Dispatcher] is the item responsible for the sorting of the incoming requests. It scans the type of request that it receives from the \textit{ViewRender} and selects the suitable component that will take care of it.
\item[ReservationManager] is the module designed to manage the reservation requests. It elaborates the data it gets from the \textit{Dispatcher} and it produces an output in terms of a Web page that it sends to the \textit{ViewRender}. In doing that, it has to communicate with the \textit{Model} component to get information from the Database and eventually to update it. It also exchanges information with the \textit{ServerCommunicationManager} to notify the selected car of the reservation request.
\item[RegistrationManager] handles the registration requests. It is connected to the \textit{Model} in order to update the data on the Database. It also produces an appropriate Web page and sends it to the \textit{ViewRender}.
\item[StateManager] is the component that takes care of updating the cars state. Thus, it is linked with the \textit{Model} and with the \textit{ServerCommunicationManager}.
As the previous Manager components, it produces an Web page to notify the success of the operation.
\item[LogInManager] deals with the log in requests. It updates the Database thanks to the \textit{Model} component. It produces a Web page. \item[MapController] is the component that manages the map that the user and the system can deal with. It makes use of an external \textit{MapService} thanks to the \textit{MapAPI} and it retrieves data (such as the location of the cars or the Safe areas and Charging areas) from the database and updates the \textit{Model} too. It exchanges information with the \textit{ReservationManager} too.
\item[RideManager] is in charge of the rides management. It has both to retrieve data from the car (thanks to the \textit{ServerCommunicationManager} and the \textit{Dispatcher})
and to ship information to it. To do so, it also needs to get info from the \textit{MapController}.\\It is linked to the \textit{RideCostCalculator} and \textit{PaymentManager} components that offers to it the obvious services.
Finally, this component access the Database through the \textit{Model}.
\item[RideCostCalculator] is the specific component that compute the cost of a ride. To perform this operation it needs info from the \textit{RideManager}.
\item[PaymentManager] handles the communication with the external \textit{Payment System}. To do so, it uses the \textit{PaymentAPI}.
\item[ServerCommunicationManager] 
is the specific component of the Webapp that manages the communication with the car. It receives info from the car thanks to the \textit{CarCommunicationManager} through the \textit{ServerAPI} and sends them to the \textit{Dispatcher} to be elaborated. It also delivers info (coming from the \textit{ReservationManager}, the \textit{StateManager} and the \textit{RideManager}) to the car exploiting the \textit{CarAPI}
\item[Model] is the component that is responsible for the data management within the app. It is linked to all the Manager components and it is the bridge between the server on which the app is running and the Database server. A suitable interface is provided by the DBMS, \textit{DBMSAPI}.
\end{description}

\noindent With regard to the Car, the following components are designed:
\begin{description}
\item[CarCommunicationManager] handles the data that the car receives from the Webapp and those data that it has do send back. It is linked to the \textit{CentralUnit}.
\item[CentralUnit] manages the info that come from all the sensors within the car and those that it receives from the server. As a consequence, it communicates with the \textit{ScreenManger} to update the data to be displayed on the screen.
\item[ScreenManager] is the component that organize the data received from the \textit{CentralUnit} and display them on the screen panel. It also gets the user input and delivers it to the \textit{CentralUnit}.
\end{description}
\imagebox{../Architecture_Overview_Img/ComponentDiagram.jpg}{Component Diagram}{0}
\section{Deployment View}
%this is what you have in slide 8, that is, the identification of the artifact that need to be deployed to have the system working
As already seen in the Overview section, the system is organised in a three-tier architecture. This section explores the distribution of the application over the physical nodes. Let us introduce the following image which represents the deployment of the application:
\imageboxNoFit{../Architecture_Overview_Img/DeploymentDiagram.jpg}{Deployment Diagram}{0}{0.5}
\noindent The \textbf{Client Tier} includes three types of devices: \textit{Mobile Device}, \textit{PC} and \textit{Car}. The first two ones can access the Webapp surfing the \textit{Internet} through a browser that is already installed on the device.
As already stated in the Overview, the \textit{Cars} are treated slight differently and they have a \textit{Built In SW} which manages the communication with the WebApp. The connection with the server passes through \textit{Internet} anyway.\\
The \textbf{Application Tier} consists in an \textit{Application Server} on which the business logic of the application is run. It is here where all the requests coming from the devices of the Client Tier
are processed and as a consequence proper Web pages are created. The \textit{Application Server} is then linked to a remote \textit{Database Server} through a LAN network.\\
The \textbf{Data Tier} hosts a \textit{Database Server}, that is a remote machine whose job is to store all the data that are necessary for the system to provide its functionalities. The \textit{Database Server} can access the physical DB through an appropriate software, \textit{MySql} which makes possible the exchange of data from and to the \textit{Application Server}.

\section{Runtime View}
%may use sequence diagrams to describe the way components interact to accomplish specific tasks typically related to use cases
%this is what you have in slide 9 plus sequence diagrams describing the way components behave in order to accomplish a certain activity
In this section, some \textit{sequence diagrams} will be provided in order to better explain how the components of the system behave and interact with each others to fulfill the key functionalities.
\subsection{Reservation}
\imagebox{../UML/UML_img/Reservation_sequence}{Reservation - Sequence Diagram}{0}
The flow of information starts from the user, through the ViewRender, and then request is dispatched to the ReservationManager. It is responsible for updating the model and to notify the operation also to the chosen car. Finally the user receive a confirmation for his reservation.
\subsection{Ride Start}
In order to pick-up an available car, the user has to use the Webapp to start the operation. Before unlocking the car, the system checks that the user balance is at least greater than zero. In the negative case, the user is notified about the abortion of the operation, caused by a non-positive balance. Otherwise the system change the status of the chosen car and unlocks it. Then the user can get on board, and through the touch screen can activate the \textit{Money Saving Option} and enter his final destination. Finally he presses the \textit{start} button. The system detects whether the user has enabled the special option and in this case it searches and notifies the suggested area where to park in order to be eligible to obtain the discount on the ride.
Finally, the on board screen prompts a message stating that the user can start driving.
\imagebox{../UML/UML_img/RideStart_sequence}{Start of the Ride -  Sequence Diagram}{0}
\subsection{Ride Stop}
The stop signal of the ride is given by the user from the touch screen embedded into the car. So the car communicates with the system, sending the message that the user has ended his ride and along with it, it sends also the key parameters (i.e. final position, battery level and people on board) that will be used to calculate the discount or the penalty that has to be applied to the ride cost. Then the system computes a temporary cost of the ride that will be displayed on the car screen. The user confirms it and gets off the car. After the confirmation, the user has a limited time to plug the car into the power grid to obtain the relative discount. When the timer ends, the car sends a message to the system to communicate whether the user has plugged in the car. At this point the system calculates the real cost of the last ride, since it has all the required data, and complete the payment stage of the operation through the external system.
\imagebox{../UML/UML_img/RideStop_sequence}{Stop of the Ride - Sequence Diagram}{0}
\subsection{Car Plug-in}
The employee of the company is responsible for plugging in all the cars that haven't been attached to the power grid from the users. Through the Webapp the employee registers the operation, then the system takes care of updating the status of the car.
\imagebox{../UML/UML_img/ReplugCar_sequence}{Car Plug In by Employee - Sequence Diagram}{0}
\section{Component interfaces}
%here you define the interfaces of your components, that is, which operations they offer to the external world, their meaning, any input and output parameter (name, possible set of values/type)
\section{Selected Architectural Styles and Patterns}
%explanation (which, why and how) styles/patterns used
In this section the design principles and choices that have been adopted in the development of the system will be faced.\\First of all, the type of the application.\textit{PowerEnJoy} will be developed as a Webapp in order to make it independent from the particular underlying Operating System. Thus, the software portability increases and the programmers' work overhead decreases.\\The choice of a Webapp leads to the implementation of a standard multi-tier architecture and this is why the proposed one is a three-tier structure. The presence of heterogeneous nodes on the \textit{Client Tier} (PCs/Smartphones and Cars) with different roles  has to be taken into account in the design phase. The solution is achieved with a proper and different software layers organisation: PCs/Smartphones run only the Presentation layer through the browser, while the Cars are equipped with a Built In SW which is in charge of both the GUI and part of the application logic. The system, however, is able to interact with both of these clients thanks to proper API.\\With regard to software design principles, there can be found, at this stage, at least two suitable design patterns: the \textit{MVC} and \textit{State} patterns.\\The \textit{MVC Pattern} is recommended for identifying the logical functionalities of the software modules (thus, increasing the cohesion) and to make it easier to apply changes to them in the future (thus, increasing the maintainability).\\The \textit{State Pattern} is suitable for treating the car states. With this solution, in each state, only a defined set of operations can be applied to a car. This makes the problem of managing a car simpler and avoids undesired behaviours.


\section{Other Design Decisions}
\subsection{ER Diagram}
An important issue in the Design Process is to select the suitable data that the system has to store in the database.This is crucial in the Design Phase because it affects the implementation of all the system functionalities, which rely on these data.The following diagram shows the pieces of information that the database server provides to the application.
\imagebox{../UML/UML_img/ERProject}{ER Diagram}{0}
\noindent The diagram suggests that the system has two types of \textit{Person}: \textit{User} and \textit{Employee}. As already stated in the \textit{RASD} this is done to distinguish the roles and permissions of the two types of actors.\\
The system has to store information about the \textit{Ride} such as the time (in order to be able to compute the cost), possible discount or extra charge and the initial and final position so to keep track of the associated \textit{Car}.\\\textit{Cars} are identified by their unique license plate and the system is always updated on their current state, which is one of the shown above.\\The system also records the info of each \textit{Reservation} in terms of the user and the corresponding car. This is clearly not enough to identify in a unique way a reservation, so the attribute Id is used as a key field for it.\\
Finally the system stores the data that are necessary to properly represent the \textit{SafeAreas} and the \textit{ChargingAreas}: their position, their id, information about the number of total outlets and the cars that are parked. This piece of information is crucial for some key functionalities of the system, such as the computation of the uniform distribution of cars in the city.

\chapter{Algorithm Design}
%Focus on the definition of the most relevant algorithmic part
\section{Discount and Penalties Calculation}
\begin{pseudocodeEnv}
//Discount calculator

float CAR_PENALITY := 0.3;
float CAR_CHARGE_DISCOUNT := 0.2;
float CAR_PEOPLE_DISCOUNT := 0.1;
float CAR_POSITION_DISCOUNT := 0.3;
float CAR_CHARGE_CONSTANT := 0.5;
float CAR_PEOPLE_CONSTANT := 0.3;
float MIN_BATTERY_CONSTANT := 0.2;

//Check if the user has some discounts or penalities
function float calculateDiscount (Position position, integer peopleNumber, float carCharge, integer price, Car car){
	if not checkPenalty()
		return price*(1-max(checkDiscount(position, car), checkDiscount(peopleNumber), checkDiscount(carCharge)));
	endif
	return price*(1+CAR_PENALITY);
}

//Check if the user can obtain the discount of the car left with more than 50% of battery
function integer checkDiscount(float carCharge){
	if carCharge >= CAR_CHARGE_CONSTANT
		return CAR_CHARGE_DISCOUNT;
	endif
	return 0;
}

//Check if the user can obtain the discount because he had two passengers
function integer checkDiscount(integer peopleNumber){
	if peopleNumber >= CAR_PEOPLE_CONSTANT
		return CAR_PEOPLE_DISCOUNT;
	endif
	return 0;
}

//Check if the user can obtain the discount because he had left the car plugged in a charging area
function integer checkDiscount(Position position, Car car){
	if chargingAreaPositions.contains(position) && car.isPlugged()
		return CAR_POSITION_DISCOUNT;
	endif
	return 0;
}

//Check if the user has left the car with left of 20% of battery or he has left the car too far from the nearest charging area
function bool checkPenalty(){
	if carCharge < MIN_BATTERY_CONSTANT || tooFarFrom(position)
		return true;
	endif
	return false;
}

//Check if exists a charging area near the car
function boolean tooFarFrom(Position position){
	for each Position called chargingAreaPosition in chargingAreaPositions)
		if Position.distance(chargingAreaPosition, position)<MAX_DISTANCE
			return false;
		endif
	endfor
	return true;
}
\end{pseudocodeEnv}
\section{Cars Uniform Distribution}
\begin{pseudocodeEnv}
//Uniform distribution	

float KM_INCREASE := 0.5;
	
//This function gives the nearest position to the place where you want to park
function Position giveBestPosition(Position finalPosition){
	int totalPositionCounted := 0;
	int totalCarCounted := 0;
	return checkUniformity(finalPosition, totalPositionCounted, totalCarCounted, KM_INCREASE).getFirst().getPosition();
}
	
//This method checks the uniformity finding the best place where the variance is high
function List<PositionMediaVariance> checkUniformity(finalPosition, totalPositionCounted, totalCarCounted, range){
	List<PositionMediaVariance> positionMediaVarianceList;
	
	if totalCarCounted == 0 && totalPositionCounted == 0
		for each Position called chargingAreaPosition in chargingAreaPosition)
			totalPositionCounted++;
			totalCarCounted = totalCarCounted + chargingAreaPosition.getCars().size();
		endfor
	endif
	
	for each Position called chargingAreaPosition in nearChargingAreaPositions(range, finalPosition){
		if chargingAreaPosition.hasFreePlug()
			positionMediaVarianceList.add(new PositionMediaVariance(chargingAreaPosition, totalCarCounted / totalPositionCounted, totalCarCounted / totalPositionCounted - chargingAreaPosition.getCars()));
		endif
	}
		
	if not positionMediaVariance.isEmpty(){
		return positionMediaVarianceList.sort();
	}
	
	checkUniformity(finalPosition, totalPositionCounted, totalCarCounted, range + KM_INCREASE);
}

//This checks the nearest charging area in a certain range.
function List<Position> nearChargingAreaPositions(int range){
	List<Position> nearChargingAreaPositionsArrayList;
	for each Position called chargingAreaPosition in chargingAreaPositions)
		if Position.distance(chargingAreaPosition, finalPosition) < range
			nearChargingAreaPositionsArrayList.add(chargingAreaPosition);
		endif
	endfor
}

//Return if the car is plugged in a charging area
function boolean isPlugged()
\end{pseudocodeEnv}
\chapter{User Interface Design}
%Provide an overview on how the user interface(s) of your system will look like; if you have included this part in the RASD, you can simply refer to what you have already done, possibly, providing here some extensions if applicable.
\imagebox{../Mockup_img/Web.png}{Webapp Interface on Personal Computers}{0}
\imagebox{../Mockup_img/Mobile.png}{Mobile Interface on Mobile Devices}{0}
\imagebox{../Mockup_img/Car.png}{On-Board Car Screen Interface}{0}
\chapter{Requirements Traceability}
%Explain how the requirements you have defined in the RASD map to the design elements that you have defined in this document.
This section shows how both Functional \& Non-Functional Requirements, already listed in the \textit{RASD}, are mapped with respect to the components that the system is made of.
\section{Functional Requirements}
	\begin{Req}
		\item The system checks whether the user entered all the required data.
			\begin{itemize}
				\item ViewRender
			\end{itemize}
		\item The system checks whether the current user is already registered or not.
			\begin{itemize}
				\item ViewRender
				\item RegistrationManager
				\item Model
			\end{itemize}
		\item The systems checks that there is only one user with that name.
			\begin{itemize}
				\item ViewRender
				\item RegistrationManager
				\item Model
			\end{itemize}
		\item The system generates a password for the access and sends it back to the new user.
			\begin{itemize}
				\item ViewRender
				\item RegistrationManager
			\end{itemize}
		\item  The system checks the input data and verifies whether the account really exists.
			\begin{itemize}
				\item ViewRender
				\item LogInManager
				\item Model
			\end{itemize}
		\item The system confirms the access.
			\begin{itemize}
				\item ViewRender
				\item LogInManager
			\end{itemize}
		\item The system verifies whether the given location exists.
			\begin{itemize}
				\item RideManager
				\item MapController
			\end{itemize}
		\item The system shows the available cars within the range of distance specified by the user from the given location.
			\begin{itemize}
				\item ViewRender
				\item MapController
			\end{itemize}
		\item The system obtains the user position via GPS.
			\begin{itemize}
				\item ViewRender
				\item Dispatcher
			\end{itemize}
		\item The system shows the available cars within the range of distance specified by the user from his current position.
			\begin{itemize}
				\item ViewRender
				\item Dispatcher
				\item MapController
				\item Model
			\end{itemize}
		\item The system obtains the user position via GPS and checks he is nearby.
			\begin{itemize}
				\item ViewRender
				\item Dispatcher
				\item MapController
				\item Model
			\end{itemize}
		\item The system verifies the car is available.
			\begin{itemize}
				\item Model
			\end{itemize}
		\item The system unlocks the car.
			\begin{itemize}
				\item RideManager
				\item Model
				\item ServerCommunicationManager
				\item CarCommunicationManager
				\item CentralUnit
			\end{itemize}
		\item The system shows those cars that are in the area specified by the user.
			\begin{itemize}
				\item ViewRender
				\item Dispatcher
				\item MapController
				\item Model
			\end{itemize}
		\item The system verifies that the user balance is not negative.
			\begin{itemize}
				\item RideManager
				\item PaymentManager
			\end{itemize}
		\item The system flags the car as Reserved.
			\begin{itemize}
				\item ReservationManger
				\item StateManager
				\item Model
			\end{itemize}
		\item The system checks that the user has one active reservation.
			\begin{itemize}
				\item ReservationManager
				\item Model
			\end{itemize}
		\item The system lets the user delete his active reservation.
			\begin{itemize}
				\item ReservationManager
				\item Model
			\end{itemize}
		\item The system checks the position of the car at the end of the ride.
			\begin{itemize}
				\item CentralUnit
				\item CarCommunicationManager
				\item ServerCommunicationManager
				\item Dispatcher
				\item RideManager
				\item MapController
			\end{itemize}
		\item The system checks the battery charge level at the end of the ride.
			\begin{itemize}
				\item CentralUnit
				\item CarCommunicationManager
				\item ServerCommunicationManager
				\item Dispatcher
				\item RideManager
			\end{itemize}
		\item The system checks whether the car is plugged into the power grid by the user at the end of the ride.
			\begin{itemize}
				\item CentralUnit
				\item CarCommunicationManager
				\item ServerCommunicationManager
				\item Dispatcher
				\item RideManager
			\end{itemize}
		\item The system checks the number of people on board.
				\begin{itemize}
				\item CentralUnit
				\item CarCommunicationManager
				\item ServerCommunicationManager
				\item Dispatcher
				\item RideManager
			\end{itemize}
		\item The system computes the discount or extra charge.
			\begin{itemize}
				\item RideManager
				\item RideCostCalculator
			\end{itemize}
		\item The system computes the total cost of the ride.
			\begin{itemize}
				\item RideManager
				\item RideCostCalculator
			\end{itemize}
		\item The system shows the cost on the car screen.
			\begin{itemize}
				\item RideManager
				\item ServerCommunicationManager
				\item CarCommunicationManager
				\item CentralUnit
				\item ScreenManager 
			\end{itemize}
		\item The system requests the payment to the external system.
			\begin{itemize}
				\item RideManager
				\item PaymentManager 
			\end{itemize}
		\item The system locks the car.
			\begin{itemize}
				\item RideManager
				\item ServerCommunicationManager
				\item CarCommunicationManager
				\item CentralUnit
			\end{itemize}
		\item The system confirms the access in administrator mode.
			\begin{itemize}
				\item ViewRender
				\item Dispatcher
				\item LogInManager
				\item Model
			\end{itemize}
		\item The system shows the cars within the range of distance specified by the employee from the given location.
			\begin{itemize}
				\item ViewRender
				\item Dispatcher
				\item MapController
				\item Model
			\end{itemize}
		\item The system shows the current state of the car.
			\begin{itemize}
				\item ViewRender
				\item StateManager
				\item Model
			\end{itemize}
		\item The system updates the state of the car with the new value.
			\begin{itemize}
				\item ViewRender
				\item Dispatcher
				\item StateManager
				\item Model
			\end{itemize}
	\end{Req}
\section{Non Functional Requirements}
\begin{enumerate}
	\item In order to distinguish between the users and the employees, the system provides two different modes of execution of the application: user mode and administrator mode. This is also graphically represented by a slightly different interface of the app.
	\begin{itemize}
		\item ViewRender
		\item LogInManager
		\item Model
	\end{itemize}
	\item The reservation has an upper limit on time: the system can reserve a car for up to one hour before the ride. After that time expires with no ride, the system charges 1 EUR fee to the user.
	\begin{itemize}
		\item ReservationManager
		\item StateManager
		\item Model
		\item PaymentManager
	\end{itemize}
	\item The system provides discounts or extra charges on the last ride with the policy specified in the \textit{RASD}.
	\begin{itemize}
		\item RideManager
		\item RideCostCalculator
	\end{itemize}
\end{enumerate}
\chapter{Effort Spent}
%In this section you will include information about the number of hours each group member has worked towards the fulfillment of this deadline.
\chapter{References}
\section{Tools}
During the writing of this document, the following application tools have been used:
\begin{itemize}
	\item Star UML, for creating all types of UML models
	\item TeXstudio, for writing the document in \LaTeX
	\item Balsamiq, for creating the mockups of the user interface
	\item draw.io, for drawing the diagrams of the architecture
\end{itemize}
%-------------------------------------------------------------------------
%   END DOCUMENT
%-------------------------------------------------------------------------
\end{document} 