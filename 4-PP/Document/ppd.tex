\documentclass[11pt,a4paper]{report}

%INPUT of SETTINGS (usepackage, styles, new command,...)
\input{ppd_prefs}

%-------------------------------------------------------------------------
%   START DOCUMENT 
%-------------------------------------------------------------------------
\begin{document}

% INPUT OF TITLE PAGE
\input{ppd_titlepage}

%-------------------------------------------------------------------------
%   TOC
%-------------------------------------------------------------------------
%\pagestyle{plain}
\pagenumbering{roman}
\thispagestyle{empty}
\tableofcontents
\listoffigures
\cleardoublepage
\pagenumbering{arabic}
\pagestyle{fancy}

%--------------------------------------------------------------------
%  BEGIN TEXT
%--------------------------------------------------------------------
\chapter{Introduction}
\section{Revision History}
\begin{tabularx}{\textwidth}{|l|l|X|}
\hline
Version & Date & Summary\\
\hline
1.0 & 22/01/2017 & First release of the document\\
\hline
\end{tabularx}
\section{Purpose and Scope}
\section{Definitions, Acronyms, Abbreviations}
\begin{description}
	\item[PowerEnJoy]is the name of the system that has to be developed.
	\item[System]sometimes called also \textit{system-to-be}, represents the application that will be described and implemented.
	In particular, its structure and implementation will be explained in the following documents. People that will use the car-sharing service will interact with it, via some interfaces, in order to complete some operations (e.g.: reservation and renting).
	\item[Renting]it is the act of picking-up an available car and of starting to drive.
	\item[Ride] the event of picking-up a car, driving through the city and parking it. Every Ride is associated to a single user and to a single car.
	\item[Reservation]it is the action of booking an available car.
	\item[Car] a car is an electrical vehicle that will be used by a registered user.
	\item[Not Registered User] indicates a person who hasn't registered to the system yet; for this reason he can't access to any of the offered function. The only possible action that he can carry out is the registration to get a personal account.
	\item[Registered User] interacts with the system to use the sharing service. He has an account (which contains personal information, driving license number and payment data) that must be used to access to the application in order to exploit all the functionalities.
	\item[Employee] it's a person who works for the company, whose main task is to plug into the power grid those cars that haven't been plugged in by the users. He is also in charge of taking care of the status of the cars and of moving the vehicles from a safe area to a charging area and vice versa if needed.
	\item[Safe Area] indicates a set of parking lots where the users have to leave the car at the end of the rent; the set of the Safe Areas is pre-defined by the system management. These areas are spread all over the city.
	\item[Plug] defines the electrical component that physically connects the car to the power grid.
	\item[Charging Area] is a special \textit{Safe Area} that also provides a certain number of plugs that connect the cars to the power grid in order to recharge the battery.
	\item[Registration] the procedure that an unregistered user has to perform to become a registered user. At the end, the unregistered user will have an account. To complete this operation three different types of data are required: personal information, driving license number and payment info.
	\item[Search] this functionality lets the registered user search for available cars within a certain range from his/her current position or from a specified address.
	\item[RASD] is the acronym of \textit{Requirements Analysis and Specification Document}
	\item[DD] is the acronym of \textit{Design Document}
	\item[ITPD] is the acronym of \textit{Integration Test Plan Document}
	\item[PPD] is the acronym of \textit{Project Plan Document}
\end{description}
\section{Reference Documents}
	\begin{itemize}
	\item Project Assignments 2016-2017
	\item RASD v1.1
	\item DD v1.1
	\item ITPD v1.0
\end{itemize}
\chapter{Project Size Estimation}
This section talks about the estimation of the size of the project, giving an estimation of lines of code that must be writter.
This evaluation comes from the evaluation of the lines of code that occure to write only the Business Logic part, leaving out the User Interface part.\\
\textit{Function Point} is the approach used to simulate how the project will be expensive in terms of lines of code.\\
From this estimation it will be easy to find how much time will be spent on developing the Business Logic part of the project.
\section{Size Estimation: function point}
Functionalities that are to be developed and their complexity are provided by the Function Point approach.\\
\\\\
\centerline{\textit{Internal Logic Files \& External Logic Files}}
\\\\
\begin{tabularx}{\textwidth}{|X|X|X|X|X|}
	\hline
	\multicolumn{2}{|X|}{} & \multicolumn{3}{|c|}{\textit{Data Element}}\\
	\hline
	\multirow{3}{*}{\textit{Record Elements}} & & 1-19 & 20-49 & 50-2222\\
	\hline
	 & 1 & Low & Low & Adw\\
	 & 2-5 & Low & Adw & High\\
	 & 6+ & Adw & High & High\\
	\hline
\end{tabularx}
\\\\
\centerline{\textit{External Outputs \& External Inquiries}}
\\\\
\begin{tabularx}{\textwidth}{|X|X|X|X|X|}
	\hline
	\multicolumn{2}{|X|}{} & \multicolumn{3}{|c|}{\textit{Data Element}}\\
	\hline
	\multirow{3}{*}{\textit{Record Elements}} & & 1-19 & 20-49 & 50-2222\\
	\hline
	& 1 & Low & Low & Adw\\
	& 2-5 & Low & Adw & High\\
	& 6+ & Adw & High & High\\
	\hline
\end{tabularx}
\\\\
\centerline{\textit{External Input}}
\\\\
\begin{tabularx}{\textwidth}{|X|X|X|X|X|}
	\hline
	\multicolumn{2}{|X|}{} & \multicolumn{3}{|c|}{\textit{Data Element}}\\
	\hline
	\multirow{3}{*}{\textit{Record Elements}} & & 1-19 & 20-49 & 50-2222\\
	\hline
	& 1 & Low & Low & Adw\\
	& 2-5 & Low & Adw & High\\
	& 6+ & Adw & High & High\\
	\hline
\end{tabularx}
\\\\
\centerline{\textit{Complexity Weights}}
\\\\
\begin{tabularx}{\textwidth}{|X|l|c|c|c|}
	\hline
	\multicolumn{2}{|X|}{} & \multicolumn{3}{X|}{\textit{Complexity Weight}}\\
	\hline
	\textit{Function Type} & Acronimous & \textit{Low} & \textit{Mid} & \textit{High}\\
	\hline
	Internal Logic Files & ILFs & 7 & 9 & 15\\
	External Logic Files & ELFs & 5 & 7 & 10\\
	External Inputs & EIs & 2 & 4 & 6\\
	External Inquiries & EQs & 4 & 5 & 7\\
	External Outputs & EOs & 2 & 3 & 4\\
	\hline
\end{tabularx}
\subsection{Internal Logic Files: ILFs}
Some functionalities store information into the PowerEnjoy system, in order to remember data and let the system use and manage them.\\
The login is one of these functionalities: the system has to store the \textit{User Data} in order to recognize him. When the data is stored, it will be composed by many fields explaining the personal information.\\
\textit{Safe Area}, \textit{Parking Area} and \textit{Car Data} are inserted by the system at each time a new station or vehicle is added into the system.\\
Differently, \textit{Registration} and \textit{Reservation} are continuously added, through a new operation on the app.
\\\\
\begin{tabularx}{\textwidth}{|X|X|X|}
	\hline
	\textit{ILFs} & \textit{Complexity} & \textit{Low}\\
	\hline
	User Data & Mid & 9\\
	Safe Area Data & Low & 7\\
	Parking Area Data & Low & 7\\
	Reservation Data & Mid & 9\\
	Car Data & Low & 7\\
	Registration Data & Avg & 9\\
	\hline
\end{tabularx}
\subsection{External Logic Files: ELFs}
The application manages the interaction and the system acquires information from the external services used.
\textit{Payment Data} and \textit{Map Data} are considered External Logic Files, because they are provided by external component.\\
Because of these components are not part of our system, PowerEnjoy will obtain information thanks to external APIs in order to link internal to external data.\\\\ 
\begin{tabularx}{\textwidth}{|X|X|X|}
	\hline
	\textit{ELFs} & \textit{Complexity} & \textit{Low}\\
	\hline
	Maps Data & Mid & 7\\
	Payment Data & Mid & 9\\
	\hline
\end{tabularx}
\subsection{External Inputs: EIs}
These sections underline the methods called by an operation of the User on our application. The action is so called a client-server action, because the system reacts to an User operation.\\
There are two different type of these changes:
\begin{itemize}
	\item \textbf{UI Action}: the user click some buttons, or insert some text on the User Interface of the application, or on the car screen.
	\item \textbf{Physical Action}: the user plugs the car in a Parking Area.
\end{itemize}
\begin{tabularx}{\textwidth}{|X|X|X|}
	\hline
	\textit{EIs} & \textit{Complexity} & \textit{Low}\\
	\hline
	Login & Mid & 4\\
	Create Reservation & High & 6\\
	Delete Reservation & Mid & 4\\
	Plug In & Low & 2\\
	Enable MoneySavingOption & Low & 2\\
	\hline
\end{tabularx}
\subsection{External Inquiries: EQs}
These operation that involves an input and an output are the \textit{Ride} actions: when it starts and when it stops.\\
Starting the ride, the user interact with the Car UI so that the Car System can communicate to the Server that the Ride is started, and the Server can change User Interface on board displaying his navigation.\\
Otherwise, the system will calculate the cost, when the user interacts with the UI and it will display temporary how much he has to pay.\\\\
\begin{tabularx}{\textwidth}{|X|X|X|}
	\hline
	\textit{EQs} & \textit{Complexity} & \textit{Low}\\
	\hline
	Ride Start & High & 7\\
	Ride Stop & High & 7\\
	\hline
\end{tabularx}
\subsection{External Outputs: EOs}
\subsection{Overall Estimation}
\chapter{Project Cost and Effort Estimation}
\chapter{Schedule}
\chapter{Resource Allocation}
\chapter{Risk Management and Mitigation}
This section deals with the possible risks that can threaten the system during its development process. These hazards can stem from a variety of sources and thus can be classified as:
\begin{itemize}
	\item \textbf{Project risks}: if they threaten the project plan, such as the project schedule. As can be seen in the following tables, the most common problem that can arise from such risks is a delay in the project release.
	\item \textbf{Technical risks}: they are related to the technical part of the project. They include a major variety of problems such as unskilled staff, flaws in the external adopted components, changes in the set of the requirements and so on.
	\item \textbf{Business risks}: they include a set of heterogeneous hazards such as budget cuts, sales falls and market policy flaws. If they become concrete, they can compromise the viability of the project.
	\item \textbf{Personnel risks}: this type of hazards deals with all the possible problems that can be met within the team of work.
\end{itemize}
Since a pro-active risk-management approach is desirable, for each category of risk a list of the most recurrent hazards will be taken into account, along with their probabilities to be faced, their levels of impact on the project and the strategies that can be adopted to mitigate them.
\section{Project risks}
    \begin{tabularx}{\textwidth}{ |p{3cm}|p{1.8cm}|p{1.85cm}|X|}
    \hline
    Risk & Probability & Impact & Strategy \\
    \hline
    Underestimated development time & Moderate & Serious & If the spare time is not sufficient, try to negotiate with the customer the possibility to have two releases, the first to the planned one and the second one as early as possible. \\ \hline
    A change in the direction of the project & Low & Moderate & Redact very precise and detailed documentation of the project so that new managers are able to handle the process. \\ \hline
    \end{tabularx}
\section{Technical risks}
    \begin{tabularx}{\textwidth}{ |p{3cm}|p{1.8cm}|p{1.85cm}|X|}
    \hline
    Risk & Probability & Impact & Strategy \\ \hline
    Difficulty in recruiting a skilled staff & Moderate & Serious & Be prepared and inclined to pay extra to find skilled people. If it is not sufficient, may consider the possibility of buying external components that are already developed. \\ \hline
    Changes in the requirements & Low & Serious & Be compliant with the requirements traced in the \textit{RASD} and in case changes occur evaluate the trade off related to the needed variations in the project. Increment on the price should come from the customer. \\ \hline
    External components have flaws & Moderate & Serious & Choose components that are on the market for a long time, so that they can be reliable. Experience with precedent projects may be an advantage in this case. \\ \hline
    Modification in the APIs of the external components & Low & Moderate & Write the code as portable as possible. Changes in the APIs should not break the system. Also plan updates in case such modifications occur. \\ \hline
    \end{tabularx}
\section{Business risks}
    \begin{tabularx}{\textwidth}{ |p{3cm}|p{1.8cm}|p{1.85cm}|X|}
    \hline
    Risk & Probability & Impact & Strategy \\ \hline
    Budget cuts during the development & Moderate & Tremendous & This kind of risk is the most critical one. In order to mitigate it, redact a document in which the financial benefits associated to the business of the project is clearly stated and try to convince managers not to apply cuts. \\ \hline
    Poor sales & Moderate & Serious & Make special discounts and offers at the launch of the product in order to achieve great popularity. In doing so, a market analysis has to be carried out . \\ \hline
    Competitors in the market & Moderate & Moderate & Choose a guide line for your product: should it be cheaper than the other or should it aim to optimality? In the launch phase also consider the possibility of making special discounts to attract clients. \\ \hline
    \end{tabularx}
\section{Personnel risks}
    \begin{tabularx}{\textwidth}{ |p{3cm}|p{1.8cm}|p{1.85cm}|X|}
    \hline
    Risk & Probability & Impact & Strategy \\ \hline
    Poor motivation & Moderate & Moderate & Try to select people that are skilled in particular fields and have experience in their work. Praise their commitment with rewards. \\ \hline
    Conflicts within the work team & Low & Moderate & Promote the cooperation of people by making a community that periodically meets and discuss on the level of the project. \\ \hline
    Key staff are ill at critical times in the project & Moderate & Serious & Consider the possibility to have several people working on common tasks in order to prevent excessive time losses. \\ \hline
    \end{tabularx}
\chapter{Effort Spent}
%TODO complete hours
In order to complete this document, each author worked for XX hours.
%-------------------------------------------------------------------------
%   END DOCUMENT
%-------------------------------------------------------------------------
\end{document} 