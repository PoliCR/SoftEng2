\documentclass[11pt,a4paper]{report}

%INPUT of SETTINGS (usepackage, styles, new command,...)
\input{ppd_prefs}

%-------------------------------------------------------------------------
%   START DOCUMENT 
%-------------------------------------------------------------------------
\begin{document}

% INPUT OF TITLE PAGE
\input{ppd_titlepage}

%-------------------------------------------------------------------------
%   TOC
%-------------------------------------------------------------------------
%\pagestyle{plain}
\pagenumbering{roman}
\thispagestyle{empty}
\tableofcontents
\listoffigures
\cleardoublepage
\pagenumbering{arabic}
\pagestyle{fancy}

%--------------------------------------------------------------------
%  BEGIN TEXT
%--------------------------------------------------------------------
\chapter{Introduction}
\section{Revision History}
\begin{tabularx}{\textwidth}{|l|l|X|}
\hline
Version & Date & Summary\\
\hline
1.0 & 22/01/2017 & First release of the document\\
\hline
\end{tabularx}
\section{Purpose and Scope}
\section{Definitions, Acronyms, Abbreviations}
\begin{description}
	\item[PowerEnJoy]is the name of the system that has to be developed.
	\item[System]sometimes called also \textit{system-to-be}, represents the application that will be described and implemented.
	In particular, its structure and implementation will be explained in the following documents. People that will use the car-sharing service will interact with it, via some interfaces, in order to complete some operations (e.g.: reservation and renting).
	\item[Renting]it is the act of picking-up an available car and of starting to drive.
	\item[Ride] the event of picking-up a car, driving through the city and parking it. Every Ride is associated to a single user and to a single car.
	\item[Reservation]it is the action of booking an available car.
	\item[Car] a car is an electrical vehicle that will be used by a registered user.
	\item[Not Registered User] indicates a person who hasn't registered to the system yet; for this reason he can't access to any of the offered function. The only possible action that he can carry out is the registration to get a personal account.
	\item[Registered User] interacts with the system to use the sharing service. He has an account (which contains personal information, driving license number and payment data) that must be used to access to the application in order to exploit all the functionalities.
	\item[Employee] it's a person who works for the company, whose main task is to plug into the power grid those cars that haven't been plugged in by the users. He is also in charge of taking care of the status of the cars and of moving the vehicles from a safe area to a charging area and vice versa if needed.
	\item[Safe Area] indicates a set of parking lots where the users have to leave the car at the end of the rent; the set of the Safe Areas is pre-defined by the system management. These areas are spread all over the city.
	\item[Plug] defines the electrical component that physically connects the car to the power grid.
	\item[Charging Area] is a special \textit{Safe Area} that also provides a certain number of plugs that connect the cars to the power grid in order to recharge the battery.
	\item[Registration] the procedure that an unregistered user has to perform to become a registered user. At the end, the unregistered user will have an account. To complete this operation three different types of data are required: personal information, driving license number and payment info.
	\item[Search] this functionality lets the registered user search for available cars within a certain range from his/her current position or from a specified address.
	\item[RASD] is the acronym of \textit{Requirements Analysis and Specification Document}
	\item[DD] is the acronym of \textit{Design Document}
	\item[ITPD] is the acronym of \textit{Integration Test Plan Document}
	\item[PPD] is the acronym of \textit{Project Plan Document}
\end{description}
\section{Reference Documents}
	\begin{itemize}
	\item Project Assignments 2016-2017
	\item RASD v1.1
	\item DD v1.1
	\item ITPD v1.0
\end{itemize}
\chapter{Project size, cost and effort estimation}
This section talks about the estimation of the size of the project, giving an estimation of lines of code that must be written.
This evaluation comes from the calculation of the lines of code that occur to write both the Business Logic part and the User Interface part.\\
\textit{Function Point} is the approach used to simulate how the project will be expensive in terms of lines of code.\\
From this estimation it will be easy to find how much time will be spent on developing the Business Logic part of the project.
\section{Size Estimation: function points}
Functionalities that are to be developed and their complexity are provided by the Function Point approach.\\In each table is reported how much load can take developing these parts: from \textit{Low}, not so much expensive, to \textit{High}, the heavier.\\
\\\\
\centerline{\textit{Internal Logic Files \& External Logic Files}}
\\\\
\begin{tabularx}{\textwidth}{|p{1cm}|X|X|X|X|}
	\hline
	\multicolumn{2}{|X|}{} & \multicolumn{3}{c|}{\textit{Data Element}}\\
	\hline
 	 \multirow{1}{*}{}& & 1-19 & 20-49 & 50+\\ \cline{2-5}	
	 \multirow{3}{*}{\rotatebox[origin=c]{90}{\parbox{10mm}{Record\\Elements}}}  & 1 & Low & Low & Adw\\
	 & 2-5 & Low & Adw & High\\
	 & 6+ & Adw & High & High\\
	\hline
\end{tabularx}
\\\\
Each line of this table underlines how it becomes more heavy the computation with the increasing of \textit{Data Elements} and \textit{Record Elements}.\\
Adding few records or some data, the program is rapidly loading, stretching the response time.
\\\\
\centerline{\textit{External Outputs \& External Inputs}}
\\\\
\begin{tabularx}{\textwidth}{|p{1cm}|X|X|X|X|}
	\hline
	\multicolumn{2}{|X|}{} & \multicolumn{3}{c|}{\textit{Data Element}}\\
	\hline
	\multirow{1}{*}{}& & 1-19 & 20-49 & 50+\\ \cline{2-5}	
	\multirow{3}{*}{\rotatebox[origin=c]{90}{\parbox{10mm}{Record\\Elements}}}  & 1 & Low & Low & Adw\\
	& 2-5 & Low & Adw & High\\
	& 6+ & Adw & High & High\\
	\hline
\end{tabularx}
\\\\
With \textit{External Outputs} and \textit{External Inputs}, the size of Data and Records must be smaller than the previous ones seen in \textit{Internal Logic Files} and \textit{External Logic Files}.\\We have a big load adding only one record, or a few data, getting worst the performance of the system.
\\\\
\centerline{\textit{External Inquiries}}
\\\\
\begin{tabularx}{\textwidth}{|p{1cm}|X|X|X|X|}
	\hline
	\multicolumn{2}{|X|}{} & \multicolumn{3}{c|}{\textit{Data Element}}\\
	\hline
	\multirow{1}{*}{}& & 1-19 & 20-49 & 50+\\ \cline{2-5}	
	\multirow{3}{*}{\rotatebox[origin=c]{90}{\parbox{10mm}{Record\\Elements}}}  & 1 & Low & Low & Adw\\
	& 2-5 & Low & Adw & High\\
	& 6+ & Adw & High & High\\
	\hline
\end{tabularx}
\\\\
The parts of External Input is the heavier, because it is an interaction between the User and the System.\\
The table is so heavier at littler variations. One record, or few data are enough to increase the performance of the system. 
\\\\
\centerline{\textit{Complexity Weights}}
\\\\
\begin{tabularx}{\textwidth}{|l|l|X|X|X|}
	\hline
	\multicolumn{2}{|X|}{} & \multicolumn{3}{c|}{\textit{Complexity Weight}}\\
	\hline
	\textit{Function Type} & Acronym & \textit{Low} & \textit{Mid} & \textit{High}\\
	\hline
	Internal Logic Files & ILFs & 7 & 9 & 15\\
	External Logic Files & ELFs & 5 & 7 & 10\\
	External Inputs & EIs & 2 & 4 & 6\\
	External Inquiries & EQs & 6 & 8 & 10\\
	External Outputs & EOs & 2 & 3 & 4\\
	\hline
\end{tabularx}
\\\\
The general complexity is so defined.\\
It is obviously heavier a Function Type like the \textit{Internal Logic Files} or an \textit{External Inquiries}, which manage the update of the Data and the communication between the User and the System, than a \textit{External Output}, \textit{External Input} or an \textit{External Logic Files}, that gives only an output or take an input.\\
\subsection{Internal Logic Files: ILFs}
Some functionalities store information into the PowerEnjoy system, in order to remember data and let the system use and manage them.\\
The login is one of these functionalities: the system has to store the \textit{User Data} in order to recognize him. When the data is stored, it will be composed by many fields explaining the personal information, in order to characterize every person.\\
\textit{Safe Area}, \textit{Parking Area} and \textit{Car Data} are inserted by the system at each time a new station or vehicle is added into the system. \textit{Safe Area} and \textit{Parking Area} will be condensed in the same table, but diversified by some attributes, in order to have an increase of performance, searching for the data in only one place. \\\\
Differently, \textit{Reservation} is continuously added, through a new operation on the app.
\\\\
\begin{tabularx}{\textwidth}{|X|X|X|}
	\hline
	\textit{ILFs} & \textit{Complexity} & \textit{Points}\\
	\hline
	User Data & Mid & 9\\
	Employee Data & Low & 7\\
	Safe Area Data & Low & 7\\
	Parking Area Data & Low & 7\\
	Reservation Data & Mid & 9\\
	Ride Data & Mid & 9\\
	Car Data & Low & 7\\
	\hline
	\hline
	Total & \multicolumn{1}{X}{} & \multicolumn{1}{X|}{55}\\
	\hline
\end{tabularx}
\subsection{External Logic Files: ELFs}
The application manages the interaction and the system acquires information from the external services used.
\textit{Payment Data} and \textit{Map Data} are considered External Logic Files, because they are provided by external component.\\
Because of these components are not part of our system, PowerEnjoy will obtain information thanks to external APIs in order to link internal to external data.\\
Payment Data will be more complex to develop, because the system has to guarantee the security of the payment and of the information, avoiding every type of external attack.\\\\
\begin{tabularx}{\textwidth}{|X|X|X|}
	\hline
	\textit{ELFs} & \textit{Complexity} & \textit{Points}\\
	\hline
	Maps Data & Mid & 7\\
	Payment Data & High & 10\\
	\hline
	\hline
	Total & \multicolumn{1}{X}{}& \multicolumn{1}{X|}{17}\\
	\hline
\end{tabularx}
\subsection{External Inputs: EIs}
These sections underline the methods called by an operation of the User on our application. The action is so called a client-server action, because the system reacts to an User operation.\\
There are two different type of these changes:
\begin{itemize}
	\item \textbf{UI Action}: the user click some buttons, or insert some text on the User Interface of the application, or on the car screen.
	\item \textbf{Physical Action}: the user plugs the car in a Parking Area.
\end{itemize}
The most complex will be the creation of the reservation, because it has to interact with data, creating them and inserting them linking the User Interface to the remote DBMS.\\It is easy to see how is more easy to develop the part that check if the car has been plugged or not, after the stop of the ride.\\\\
\begin{tabularx}{\textwidth}{|X|X|X|}
	\hline
	\textit{EIs} & \textit{Complexity} & \textit{Points}\\
	\hline
	Login & Mid & 4\\
	Logout & Low & 2 \\
	Registration & Mid & 4\\
	Insert Reservation & High & 6\\
	Delete Reservation & Mid & 4\\
	Car Plug In & High & 6\\
	Update Car Status & Low & 2\\
	Insert a new car & Mid & 4 \\
	Update a car & Mid & 4\\
	Delete a car & Mid & 4\\
	Insert a new charging area & High & 6\\
	Update charging area & Mid & 4\\
	Delete charging area & Mid & 4\\
	Convey end ride via touch screen & Mid & 4 \\ 
	Insert final destination & Mid & 4\\
	\hline
	\hline
	Total & \multicolumn{1}{X}{} & \multicolumn{1}{X|}{62}\\
	\hline
\end{tabularx}
\subsection{External Inquiries: EQs}
These operation that involves an input and an output are the \textit{Ride} actions: when it starts and when it stops.\\
Starting the ride, the user interact with the Car UI so that the Car System can communicate to the Server that the Ride is started, and the Server can change User Interface on board displaying his navigation.\\
Otherwise, the system will calculate the cost, when the user interacts with the UI and it will display temporary how much he has to pay.\\
So \textit{Ride Start} and \textit{Ride Stop} are easily considered as high complex to develop, because they have different communication between the car and the system.\\
Differently, the \textit{Search} for a car or the \textit{Pick Up} of it, are actions which involve only one communication car - system, so they are considered easier to develop.\\\\
\begin{tabularx}{\textwidth}{|X|X|X|}
	\hline
	\textit{EQs} & \textit{Complexity} & \textit{Points}\\
	\hline
	Search for available cars & Mid & 8\\
	Show cars information & Mid & 8\\
	Enable MoneySavingOption & High & 10\\
	Pick Up a car & High & 10\\
	Ride initialization & HIgh & 8\\
	\hline
	\hline
	Total & \multicolumn{1}{X}{} & \multicolumn{1}{X|}{44}\\
	\hline
\end{tabularx}
\subsection{External Outputs: EOs}
External Outputs are elementary operations that generates data that are presented to the User, without any client action.\\
The cost notification is one example from EOs section: the user, stopping the ride, gets out of the car, and after a certain time the system checks if he has plugged or not his vehicle. After this, the system displays how much the user has to pay.\\ This is an example of an operation send by the system without any actions of the user.
\\\\
\begin{tabularx}{\textwidth}{|X|X|X|}
	\hline
	\textit{EOs} & \textit{Complexity} & \textit{Points}\\
	\hline
	Payment Notification & Low & 2\\
	Notify Expired Reservation & Mid & 4\\
	\hline
	\hline
	Total & \multicolumn{1}{X}{} & \multicolumn{1}{X|}{6}\\
	\hline
\end{tabularx}
\subsection{Overall Estimation}
Considering the whole estimation of the Function Types, the total will be the summation of each one of the Total previously found. 
\\\\
\begin{tabularx}{\textwidth}{|X|X|X|}
	\hline
	\textit{Function Type} & \textit{Acronym} & \textit{Value}\\
	\hline
	Internal Logic Files & ILFs & 55\\
	\hline
	External Logic Files & ELFs & 17\\
	\hline
	External Inputs & EIs & 62\\
	\hline
	External Inquiries & EQs & 44\\
	\hline
	External Outputs & EOs & 6\\
	\hline
	\hline
	Total & \multicolumn{1}{X}{} & \multicolumn{1}{X|}{184}\\
	\hline
\end{tabularx}
\\\\
Having this result, the total lines of code are easily calculable using   \textit{conversion factor} and the \textit{usage percentage} used to specify how much the project is divided in different languages. Using this method we can consider the entire project, from the Business Logic to the UI development part, adopting a weighted average between languages:
\\\\
\begin{tabularx}{\textwidth}{|X|X|X|}
	\hline
	\textbf{Language} & \textbf{Conversion factor (SLOC/FP)} & \textbf{Usage percentage}\\
	\hline
	\textit{HTML} & 34 & 15\%\\
	\hline
	\textit{JEE} & 46 & 75\%\\
	\hline
	\textit{Javascript} & 43 & 10\%\\
	\hline
\end{tabularx}
\\\\
According to these constants and the usage percentage of the language, he total lines of code to write are:\\
\begin{equation}
	\centerline{\textit{Code Lines =} $H\textsubscript{t}\cdot$($C\textsubscript{HTML}\cdot$P\textsubscript{HTML} + $C\textsubscript{JEE}\cdot$P\textsubscript{JEE} + $C\textsubscript{JS}\cdot$P\textsubscript{JS})}
\end{equation}
Where:
\begin{itemize}
	\item C\textsubscript{i} express the \textit{conversion factor} of the \textit{i} language
	\item P\textsubscript{i}
	express the \textit{usage percentage} of the \textit{i} of the language in the whole project
	\item H\textsubscript{t} is the \textit{FPs result}
\end{itemize}
The result is so found:
\begin{equation}
\centerline{\textit{Code Lines =} $184\cdot$($34\textsubscript{HTML}\cdot$0.15\textsubscript{HTML} + $46\textsubscript{JEE}\cdot$0.75\textsubscript{JEE} + $43\textsubscript{JS}\cdot$0.1\textsubscript{JS}) \textit{= 8078}}
\end{equation}
\newpage
\section{Cost and effort estimation: COCOMO II}
This section deals with the estimation of the cost and the effort that are required to develop the system. The evaluation will be carried out using the \textit{COnstructive COst MOdel II} (later on referred as \textit{COCOMO II}) and will make use of the estimated size of the project obtained in the previous section. Let us first introduce the empirical formula that will guide the effort estimation process:
\begin{equation} \label{formula:1}
Effort = A \cdot Size^E \cdot \begin{matrix} \prod_{i=1}^{17} EM_i \end{matrix}
\end{equation}
\noindent where:
\begin{itemize}
	\item Effort is expressed in PM (Person-Months)
	\item $A = 2.94$ is a productivity constant in PM/KSLOC (Person-Months/Kilo-Source Lines of Code)
	\item Size is the estimated size of the project in KSLOC obtained in the previous analysis
	\item E is an aggregated index of five \textit{Scale Factors}
	\item $EM_{i}$ stands for \textit{Effort Multiplier} and corresponds to the value of a specific entry that will be explained later in the \textit{Cost Drivers} subsection
\end{itemize}
Next subsections are focused on the evaluation of the two last operands of the previous formula. At the end of the section the final result obtained applying formula \ref{formula:1} is shown.
\subsection{Scale Factors}
The first step consists in evaluating the actual value of exponent E in formula \ref{formula:1}. To do so, a list of \textit{Scale Factors} have to be computed. These entries represent an heterogeneous set of variables and their values are assigned according to the following table:
\\\\
\begin{tabularx}{\textwidth}{ |p{1.6cm}|p{1.8cm}|p{1.6cm}|p{1.6cm}|p{1.6cm}|p{1.6cm}|X|}
    \hline
    Scale Factor & Very low & Low & Nominal & High & Very high & Extra high \\ \hline
    PREC \newline \newline \newline \newline $SF_{i}$ & Thoroughly unprecedented \newline \newline 6.20& Largely unprecedented \newline \newline 4.96& Somewhat unprecedented \newline \newline 3.72& Generally familiar \newline \newline \newline 2.48& Largely familiar \newline \newline \newline 1.24& Thoroughly  familiar \newline \newline \newline 0.00\\ \hline
    FLEX \newline \newline \newline \newline $SF_{i}$ & Rigorous \newline \newline \newline \newline 5.07& Occasional relaxation \newline \newline 4.05& Some relaxation\newline \newline \newline 3.04& General conformity \newline \newline 2.03& Some conformity \newline \newline 1.01& General goals \newline \newline \newline 0.00\\ \hline
    RESL \newline \newline \newline \newline $SF_{i}$ & Little ($20\%$) \newline \newline \newline 7.07& Some ($40\%$) \newline \newline \newline 5.65& Often ($60\%$) \newline \newline \newline 4.24& Generally ($75\%$) \newline \newline \newline 2.83& Mostly ($90\%$) \newline \newline \newline 1.41& Full ($100\%$) \newline \newline \newline \newline 0.00\\ \hline
     TEAM \newline \newline \newline \newline \newline $SF_{i}$ & Very difficult interactions \newline \newline 5.48& Some difficult interactions \newline \newline 4.38& Basically cooperative interactions \newline 3.29& Largely cooperative \newline \newline \newline 2.19& Highly cooperative \newline \newline \newline 1.10& Seamless interactions \newline \newline \newline \newline 0.00\\ \hline
      PMAT \newline \newline \newline \newline $SF_{i}$ & Level 1 Lower \newline \newline \newline 7.80& Level 1 Upper \newline \newline \newline 6.24& Level 2 \newline \newline \newline \newline 4.68& Level 3 \newline \newline \newline \newline 3.12& Level 4 \newline \newline \newline \newline 1.56& Level 5 \newline \newline \newline \newline 0.00\\ \hline
\end{tabularx}\\\\\\
Let us discuss each of these entries:
\begin{itemize}
	\item \textbf{Precedentedness}: It is the index that defines the previous experience of the team in developing such projects. This is the very first project, thus a \textit{Very low} level is reasonable.
	\item \textbf{Development flexibility}: The index of the flexibility for the project choices with respect to the external interfaces and the requirements. The project has strict requirements to abide by in terms of functionalities of the system, but no specific architecture is required. This variable is then assigned a level of \textit{Nominal}.
	\item \textbf{Architecture and risk resolution}: It reflects the capability of managing possible risks with an accurate contingency plan and the fact that a reasonable schedule is sketched. In the document both these aspects are covered in detail, so for this variable the right choice is to pick the \textit{Very High} level.
	\item \textbf{Team cohesion}: It deals with the capability of the members of the team to work together in harmony and sharing the same vision. The team is made of only three people in this project and they work with great commitment and together, thus \textit{Very high} is the right degree.
	\item \textbf{Process Maturity}: it is a index of the maturity of the organization. Even though this is the very first project of the team, the schedule has been accurately defined, the project widely documented and discussed among the members of the team. Consequently \textit{Nominal} level is reasonable.
\end{itemize}
Here is the re-cap of the \textit{Scale Factors} analysis:
\\\\
\begin{tabularx}{\textwidth}{|p{8cm}|X|X|}
	\hline
	\textit{Scale Factor} & \textit{Level} & \textit{Value}\\
	\hline
	Precedentness (PREC) & Very low & 6.20\\
	Development flexibility (FLEX) & Nominal & 3.04\\
	Architecture and risk resolution (RESL) & Very high & 1.41\\
	Team cohesion (TEAM) & Very high & 1.10\\
	Process Maturity (PMAT) & Nominal & 4.68\\
	\hline
	Total &  & 16.43\\
	\hline
\end{tabularx}\\\\\\
It is now possible to evaluate exponent $E$ with the following formula:
$$E = 0.91+0.01 \cdot \sum_{i=1}^5 SF_i$$
The result is 
$$E = 1.0743$$
\subsection{Cost Drivers}
\chapter{Schedule}
A fundamental part of the project plan is also the scheduling. This step allows to divide the overall work that has to be done in separated tasks and moreover to define when and how they will be completed; in particular the estimation of the calendar time required to execute each task is provided through the Gantt diagram.\\
It should be noted that the time estimation of the tasks is not trivial and while doing this it's important to consider that the productivity is not proportional to the number of people of the team and that some extra time must be added because sometimes the flow of the events does not follow the expected one.\\
The analysis and the definition of the various tasks is carried out at a quite coarse grained level thus only the main activities are included into the Gantt diagram, however they shouldn't be too small (for example a reasonable length is between 1 and 2 weeks).\\
Lastly, the start of the schedule of the project is set to the month of October.\\
\chapter{Resource Allocation}
The Resource Allocation phase is strongly correlated with the Schedule phase because its main goal is to properly assign every single task, defined in the previous step, to the components of the development team.
The assignment process should be done carefully and in such a way to optimize the effort of each team member; for instance, the allocation of all the activities and tasks is done with respect to the three authors of these documents assuming that will be also the development team.\\
As for the Schedule, the Resource Allocation is presented via a Gantt diagram too.\\

\chapter{Risk Management and Mitigation}
This section deals with the possible risks that can threaten the system during its development process. These hazards can stem from a variety of sources and thus can be classified as:
\begin{itemize}
	\item \textbf{Project risks}: if they threaten the project plan, such as the project schedule. As can be seen in the following tables, the most common problem that can arise from such risks is a delay in the project release.
	\item \textbf{Technical risks}: they are related to the technical part of the project. They include a major variety of problems such as unskilled staff, flaws in the external adopted components, changes in the set of the requirements and so on.
	\item \textbf{Business risks}: they include a set of heterogeneous hazards such as budget cuts, sales falls and market policy flaws. If they become concrete, they can compromise the viability of the project.
	\item \textbf{Personnel risks}: this type of hazards deals with all the possible problems that can be met within the team of work.
\end{itemize}
Since a pro-active risk-management approach is desirable, for each category of risk a list of the most recurrent hazards will be taken into account, along with their probabilities to be faced, their levels of impact on the project and the strategies that can be adopted to mitigate them.
\section{Project risks}
    \begin{tabularx}{\textwidth}{ |p{3cm}|p{1.8cm}|p{1.85cm}|X|}
    \hline
    Risk & Probability & Impact & Strategy \\
    \hline
    Underestimated development time & Moderate & Serious & If the spare time is not sufficient, try to negotiate with the customer the possibility to have two releases, the first to the planned one and the second one as early as possible. \\ \hline
    A change in the direction of the project & Low & Moderate & Redact very precise and detailed documentation of the project so that new managers are able to handle the process. \\ \hline
    \end{tabularx}
\section{Technical risks}
    \begin{tabularx}{\textwidth}{ |p{3cm}|p{1.8cm}|p{1.85cm}|X|}
    \hline
    Risk & Probability & Impact & Strategy \\ \hline
    Difficulty in recruiting a skilled staff & Moderate & Serious & Be prepared and inclined to pay extra to find skilled people. If it is not sufficient, may consider the possibility of buying external components that are already developed. \\ \hline
    Changes in the requirements & Low & Serious & Be compliant with the requirements traced in the \textit{RASD} and in case changes occur evaluate the trade off related to the needed variations in the project. Increment on the price should come from the customer. \\ \hline
    External components have flaws & Moderate & Serious & Choose components that are on the market for a long time, so that they can be reliable. Experience with precedent projects may be an advantage in this case. \\ \hline
    Modification in the APIs of the external components & Low & Moderate & Write the code as portable as possible. Changes in the APIs should not break the system. Also plan updates in case such modifications occur. \\ \hline
    \end{tabularx}
\section{Business risks}
    \begin{tabularx}{\textwidth}{ |p{3cm}|p{1.8cm}|p{1.85cm}|X|}
    \hline
    Risk & Probability & Impact & Strategy \\ \hline
    Budget cuts during the development & Moderate & Tremendous & This kind of risk is the most critical one. In order to mitigate it, redact a document in which the financial benefits associated to the business of the project is clearly stated and try to convince managers not to apply cuts. \\ \hline
    Poor sales & Moderate & Serious & Make special discounts and offers at the launch of the product in order to achieve great popularity. In doing so, a market analysis has to be carried out . \\ \hline
    Competitors in the market & Moderate & Moderate & Choose a guide line for your product: should it be cheaper than the other or should it aim to optimality? In the launch phase also consider the possibility of making special discounts to attract clients. \\ \hline
    \end{tabularx}
\section{Personnel risks}
    \begin{tabularx}{\textwidth}{ |p{3cm}|p{1.8cm}|p{1.85cm}|X|}
    \hline
    Risk & Probability & Impact & Strategy \\ \hline
    Poor motivation & Moderate & Moderate & Try to select people that are skilled in particular fields and have experience in their work. Praise their commitment with rewards. \\ \hline
    Conflicts within the work team & Low & Moderate & Promote the cooperation of people by making a community that periodically meets and discuss on the level of the project. \\ \hline
    Key staff are ill at critical times in the project & Moderate & Serious & Consider the possibility to have several people working on common tasks in order to prevent excessive time losses.\\
    \hline
    \end{tabularx}
\chapter{Effort Spent}
%TODO complete hours
In order to complete this document, each author worked for XX hours.
%-------------------------------------------------------------------------
%   END DOCUMENT
%-------------------------------------------------------------------------
\end{document} 