\documentclass[11pt,a4paper]{report}

%INPUT of SETTINGS (usepackage, styles, new command,...)
\input{cid_prefs}

%-------------------------------------------------------------------------
%   START DOCUMENT 
%-------------------------------------------------------------------------
\begin{document}

% INPUT OF TITLE PAGE
\input{cid_titlepage}

%-------------------------------------------------------------------------
%   TOC
%-------------------------------------------------------------------------
%\pagestyle{plain}
\pagenumbering{roman}
\thispagestyle{empty}
\tableofcontents
\cleardoublepage
\pagenumbering{arabic}
\pagestyle{fancy}

%--------------------------------------------------------------------
%  BEGIN TEXT
%--------------------------------------------------------------------
\chapter{Code description}
\section{Assigned class}
The assigned class, named \textbf{\textit{ShoppingListEvents}}, is part of the \textit{Apache OFBiz} project which focuses on enterprises processes and includes specific frameworks for ERP, CRM and other business-oriented services;
in particular the interested class is part of the \\ \textit{\textbf{org.apache.ofbiz.order.shoppinglist} package}.\\
The Code Inspection activity will be carried on the previously stated class with the intention of discovering any possible error that affects the quality and the functional behaviour of this piece of software.
\section{Functional role}
The class \textbf{\textit{ShoppingListEvents}} has to be considered in the field of orders management. As stated in the Apache OFBiz\textsuperscript{\textregistered} Project Overview \footnote{\url{https://ofbiz.apache.org/apache-ofbiz-project-overview.html}}:"The Order entities are used to manage information about sales and purchase orders and information leading up to an order".In particular the given class is in charge of managing a specific aspect of orders: the control of shopping lists and carts. As such, the class offers some suitable methods that other classes and components can use. These are the most significant:
\begin{itemize}
	\item \textit{public static String addBulkFromCart(...)} receive an \textit{HttpServletRequest} request for the creation of a shopping list along with an identifier for the list itself.
	\item \textit{public static String addListToCart(...)} scans a shopping list and add the corresponding items into the cart.
	\item \textit{public static String replaceShoppingListItem(...)} updates the shopping list with a new item.
	\item \textit{public static String getAutoSaveListId(...)} creates a special shopping list that records shopping bag contents between user visits.
	\item \textit{public static void fillAutoSaveList(...)} fills the special shopping list with the current cart.
	\item \textit{public static String saveCartToAutoSaveList(...)} saves the cart associated to the input \textit{HttpServletRequest} request to the special shopping list.
	\item \textit{public static int clearListInfo(...)} simply removes all the items from the input shopping list.
	\item \textit{public static int makeListItemSurveyResp(...)} creates records for survey responses on survey items.
	\item \textit{public static Map < String, List < String > > getItemSurveyInfos(...)} creates a  map keyed on item ID containing a list of survey response IDs.
	\item \textit{public static List < String > getItemSurveyInfo(...)} returns a list of survey response IDs for a shopping list item.
	\item \textit{public static String createGuestShoppingListCookies(...)} creates the guest cookies associated to the input \textit{HttpServletRequest} request for a shopping list.
	\item \textit{public static String clearGuestShoppingListCookies(...)} clears the guest cookies associated to the input \textit{HttpServletRequest} request for a shopping list. 
\end{itemize}
Please notice that most of the methods in this class are declared as \textit{public static}: this choice is consistent with the functional role of the class as \textit{a support for orders management}. It is likely that other classes and components use these methods to make small operations in a generic and standard way: thus, a \textit{static} access modifier is reasonable.
\chapter{Code issues}
This chapter contains the list of all the issues found in the class with respect to the check list provided within the \textit{Code Inspection Assignment Task Description} document. The first section just shows the adopted notation for reporting the various issues, while the second one, the core of this document, goes methodically through every single point of the check list and if a problem of that kind is found, using a table, the line(s) of the code affected and a brief comment are reported.
\section{Notation}
The notation adopted is simple: if no issues of a specific type are found then just a short comment is provided to motivate the absence of that problem, otherwise with the help of a table, the line(s) where the issues have been found along with a related short explanation are provided.\\
Moreover, in order to keep the same structure of the given check list and to make it possible to easily search for a specific type of issue, each point is analysed in a different subsection.
\section{Checklist issues}
\subsection{Naming Conventions}
\begin{tabularx}{\textwidth}{|l|X|}
	\hline
	\textit{Line} & \textit{Issue}\\
	\hline \hline
	602 & \multirow{3}{*}{Variables without meaningful names}\\
	610 & \\
	611 & \\
	\hline
	67 & \multirow{2}{\linewidth}{{Constants are not declared using all upper case with words separated by an underscore}}\\
	68 & \\
	\hline
\end{tabularx}
\\\\
Other variables names, methods names and class name are used properly and have meaningful names.\\Only \textit{throwaway} are sometimes composed by a one-character word.\\Class name is written with the first letter in capitalized and method names are verbs with the first letter of each subsequent word capitalized.\\Variables, methods and class are written with the camel notation.\\The other constant is written using all upper case with words separated by an underscore.
\subsection{Indention}
%\begin{tabularx}{\textwidth}{|l|X|}
%	\hline
%	\textit{Line} & \textit{Issue}\\
%	\hline
%\end{tabularx}
Throughout all the class, four spaces are used consistently to properly indent the code instructions; moreover, any \textit{tabs} command are used to indent the lines.
\subsection{Braces}
No significant issues are found with respect to bracing usage. The "Kernighan and Ritchie" style is adopted and it is consistent throughout the entire class. In general, curly braces are used for blocks within \textit{If}, \textit{while}, \textit{do-while}, \textit{try-catch} and \textit{for} clauses with only one statement too with the following exception:
\\\\
\begin{tabularx}{\textwidth}{|l|X|}
	\hline
	\textit{Line} & \textit{Issue}\\
	\hline
	\hline
	337 & \multirow{1}{*}{\textit{If} clause with only one statement is devoid of curly braces}\\
	\hline
\end{tabularx}
\subsection{File Organization}
%TODO
Blank lines and optional comments are properly used to separate different sections of the code.\\
The following table reports the lines that exceed the maximum length of 80 characters, along with a rationale.\\
\\
\begin{tabularx}{\textwidth}{|l|X|}
	\hline
	\hline
	\textit{Line} & \textit{Issue}\\
	\hline
\end{tabularx}\\
\\
Similarly to the previous table, the following one reports the lines that exceed the maximum length of 120 characters, along with a brief rationale.\\
\\
\begin{tabularx}{\textwidth}{|p{1.5 cm}|X|}
	\hline
	\textit{Line} & \textit{Issue}\\
	\hline
	\hline
	85, 99, 107, 110, 126, 151, 161, 206, 214, 216, 236, 283, 294, 299, 304, 342, 344, 369, 402, 479, 503, 656 &All these lines are characterized by an object assignment with a method call, which in most cases has a long list of parameters\\
	\hline
	95, 201, 360, 548 & In this case the lines refer to method signatures (composed of qualifiers, name, parameters, exceptions)\\
	\hline
	143, 148, 188, 289, 291, 297, 371, 414, 524 & The lines represent simply method calls and the maximum allowed length is broken because of long list of parameters. In particular, in \textit{line 188} is also present an in-line if which lengthens the numbers furthermore the length \\
	\hline 
	277 & This line refers to an if clause which is quite long because are used different method calls\\ 
	\hline
	490 & \multirow{2}{\linewidth}{These are comment lines simply explaining an else branch of an if statement}\\
	491 & \\
	\hline
\end{tabularx}
\subsection{Wrapping Lines}
In general the code is neat, tidy and statements on consecutive lines are properly aligned. Typically line breaks after commas (such as in definition of the input parameters of methods) and operators are respected with this exception:
\\\\
\begin{tabularx}{\textwidth}{|l|X|}
	\hline
	\textit{Line} & \textit{Issue}\\
	\hline
	\hline
	110 & \multirow{12}{\linewidth}{In method \textit{UtilProperties.getMessage(...)} the second parameter is attached to a comma}\\
	126 & \\
	143 & \\
	161 & \\
	206 & \\
	216 & \\
	236 & \\
	242 & \\
	294 & \\
	297 & \\
	299 & \\
	304 & \\
	\hline
\end{tabularx}
\subsection{Comments}
\begin{tabularx}{\textwidth}{|l|X|}
	\hline
	\textit{Line} & \textit{Issue}\\
	\hline
	\hline
	204 & \multirow{5}{\linewidth}{Comments used doesn't explain anything more than what the code says}\\
	240 & \\
	287 & \\
	313 & \\
	314 & \\
	\hline
\end{tabularx}
\\\\
The remaining part of the code is not commented sufficiently: there could be more lines which explain the function of a code block, especially in the critical parts.\\The other lines of code are meaningful and explain what the code is doing correctly.
\subsection{Java Source File}
%\begin{tabularx}{\textwidth}{|l|X|}
%	\hline
%	\textit{Line} & \textit{Issue}\\
%	\hline
%\end{tabularx}
The Java Source File contains only a single \textit{public class} which is, obviously, the assigned \textit{ShoppingListEvents} one. For this reason, the unique public class is also the first one in the file.\\
Since there is no complete \textit{javadoc} of the class (except for a few words that do not add any extra information with respect to the class name) it is impossible to check if the external interfaces are implemented properly because their descriptions in the javadoc is totally missing.\\
Furthermore, the javadoc of the methods is absent most times while in the case it is present, it is very short and never covers the input parameters, the returned output and the exceptions thrown by the method.
\subsection{Package and Import Statements}
No issues of this type. Package declaration (\textit{package org.apache.ofbiz.order.shoppinglist;}) is the very first non comment statement and \textit{import} statements follow.
\subsection{Class and Interface Declarations}
\begin{tabularx}{\textwidth}{|l|X|}
	\hline
	\textit{Line} & \textit{Issue}\\
	\hline
	\hline
	393 & \multirow{3}{\linewidth}{Methods aren't grouped by functionality, scope or accessibility to the other close to it}\\
	621 & \\
	684 & \\
	\hline
	95 & \multirow{3}{\linewidth}{Methods are long. They should be divided into more sub-methods that are maybe useful to reuse in future}\\
	201 & \\
	443 & \\
	\hline
\end{tabularx}\\
\\
Other class, variables and methods positions are respected and are grouped by functionality rather than by scope or accessibility.\\Moreover, the code doesn't contain any duplicates.
\subsection{Initialization and Declarations}
No critical issues are found in this field. In particular, variables and class members type are consistent with respect to their declarations.\\The scope of variables is aligned with the purpose of the block of code in which the variables are declared: since the class in consideration is essentially a list of methods for almost every variable the scope is limited to the method in which it is declared.\\Objects are always initialized before use or if a computation is needed they are set to \textit{null}.\\Most times declarations of variables occur at the beginning of the blocks in which they will be used, thus making possible to easily look up for a particular variable.\\
Event though the overall class is well structured with respect to initializations and declarations, one may question about these arguments:
\\\\
\begin{tabularx}{\textwidth}{|l|X|}
	\hline
	\textit{Line} & \textit{Issue}\\
	\hline
	\hline
	67 & \multirow{3}{\linewidth}{The three \textit{public static final} variables may be set to \textit{private} since they are unlikely to be used outside this class}\\
	68 & \\
	69 & \\
	\hline
	81 & \multirow{3}{\linewidth}{String variables \textit{shoppingListId} and \textit{selectedCartItems} are reassigned after initialization in previous lines. A legal operation, but a new object is created.}\\
	85 & \\
	121 & \\
	\hline
	293 & \multirow{4}{\linewidth}{Variables declarations should be put at the beginning of the corresponding blocks and not in these positions}\\
	332 & \\
	365 & \\
	366 & \\
	\hline
\end{tabularx}
\subsection{Method Calls}
There is no error to underline.\\Every method has parameters presented in the correct order.\\When a method is called, it is called the right one, although there are similar names.\\Return value of the method is used properly in each case. 
\subsection{Arrays}
Arrays problems is not an issue in this class since there are no off-by-one errors (i.e. array elements are accessed without indexing problems and loops are executed a right number of times) and out-of-bounds elements.
\subsection{Object Comparison}
Every object comparison is done with the \textit{Java} method \textbf{equals()} and not with the \textbf{==} or the \textbf{!=}.\\The object is correctly compared to something with the \textit{==} when it is important to see if the variable is null.
\subsection{Output Format}
In the code there aren't instructions that directly display text and messages to the users (such as, for example, \textit{System.out.println(..)}); however the code in some cases deals with variables of type \textit{String}: for each of them the strings are correctly spelled and free of grammatical errors. Also, they are always correctly formatted, in terms of both line stepping and spacing.\\
More specifically, error messages (some of them write into the log others instead are simply returned as strings) are syntactically correct, however in some cases they are not totally clear and does not provide any specific guidance on how to correct the problem. This situation can be found at the following lines:\\
\\
\begin{tabularx}{\textwidth}{|l|X|}
	\hline
	\textit{Line} & \textit{Issue}\\
	\hline
	\hline
		109 & \multirow{3}{\linewidth}{These instructions write into the log just a generic error message, which seems to be poorly detailed in order to provide a precise way on how to correct the problem}\\
		160 & \\
		235 & \\
	\hline
\end{tabularx}

\subsection{Computation, Comparisons and Assignments}
%TODO
%\begin{tabularx}{\textwidth}{|l|X|}
%	\hline
%	\textit{Line} & \textit{Issue}\\
%	\hline
%\end{tabularx}
The class implementations is devoid of any kind of \textit{brutish programming}.
\subsection{Exceptions}
\begin{tabularx}{\textwidth}{|l|X|}
	\hline
	\textit{Line} & \textit{Issue}\\
	\hline
	\hline
	142 & \multirow{2}{*}{Exception used isn't meaningful}\\
	328 & \\
	\hline
\end{tabularx}\\
\\
Other exceptions are used correctly and have a correct meaning that helps to find the problem of different program routine.
\subsection{Flow of Control}
The class has no particular complex control flow structures but is indeed quite redundant with \textit{if} statements that weigh down the reading of the code. Typically these \textit{if} clauses are not followed by \textit{else} counterparts and so the default branches in these cases are trivially the instructions that follow the \textit{if} block.\\For what concerns loops most of the times the preferred pattern is the \textit{for-each} loop, consequently initialization,increment and termination are not issues. The standard \textit{for} loops are well formed too.
\subsection{Files}
%\begin{tabularx}{\textwidth}{|l|X|}
%	\hline
%	\textit{Line} & \textit{Issue}\\
%	\hline
%\end{tabularx}
The \textit{ShoppingListEvents} class doesn't use any external file, for this reason there aren't instructions involving file management.
\section{Other issues}
This section contains problems, issues or simply remarks that are not systematically specified in the \textit{Check list issues} but can be of concern as well.\\\\
\begin{tabularx}{\textwidth}{|l|X|}
	\hline
	\textit{Line} & \textit{Issue}\\
	\hline
	\hline
	%TODO sistemare....
	285 & \multirow{2}{\linewidth}{TODO reminders must be commented with the usual notation \textit{\textbackslash\textbackslash TODO} so that every IDE can find them easily and recognize them}\\
	367 & \\
	\hline
	360 & It is better to have parameters of the same type one close to the other. At this line, \textit{String} input parameters are divided\\
	\hline
	694 & Eliminate wrapping lines that aren't used to visualize the code better\\
	\hline
	314 & The overloaded method \textit{public static String addListToCart(...)} should return a string, but this implementation of the method is used only to make small computations and it returns a non significant value (an empty string).\\
	\hline
\end{tabularx}
\chapter{Effort Spent}
%TODO completare ore lavoro
In order to complete this document, each author worked for XX hours.
\chapter{Revision History}
\begin{tabularx}{\textwidth}{|l|l|X|}
	\hline
	Version & Date & Summary\\
	\hline
	\hline
	1.0 & 05/02/2017 & First release of the CI document\\
	\hline
\end{tabularx}
%-------------------------------------------------------------------------
%   END DOCUMENT
%-------------------------------------------------------------------------
\end{document} 